\documentclass{article}
\usepackage[utf8]{inputenc}
\usepackage{amsmath}
\usepackage{algorithm}
\usepackage{algpseudocode}
\usepackage{hyperref}
\usepackage{lipsum}
\usepackage[margin=2.2cm]{geometry}

\title{Marble Slab Subdivision Algorithm}
\title{Theory Assignment-3: ADA Winter-2024}
\author{Adarsh Jha (2022024) \and Akshat Kothari (2022053)}

\date{18 February 2024}
\begin{document}

\maketitle

\section{Introduction}
You have mined a large slab of marble from a quarry, and you want to cut it into smaller rectangles to maximize profit. The marble slab has dimensions $n$ centimeters in height and $m$ centimeters in width. Spot prices $P[x, y]$ can be queried for any $x$ cm by $y$ cm marble rectangle. The goal is to describe an algorithm that subdivides the marble slab into integral pieces to maximize profit.

\section{Algorithm: COMPUTE-M}
\begin{algorithm}
\caption{COMPUTE-M Algorithm}
\begin{algorithmic}[1]
\Function{COMPUTE-M}{$n, m, p, M$}
    \If{$M[n][m] \neq \text{NIL}$}
        \State \Return \Comment{Avoid recomputing}
    \ElsIf{$n = 0$ or $m = 0$}
        \State $M[n][m] \gets 0$ \Comment{Base Case}
    \Else
        \State $max \gets 0$
        \For{$i \gets 1$ to $\lfloor n/2 \rfloor$} \Comment{Check all horizontal cuts}
            \State \Call{COMPUTE-M}{$i, m, p, M$}
            \State \Call{COMPUTE-M}{$n-i, m, p, M$}
            \If{$M[i][m] + M[n-i][m] > \text{max}$}
                \State $max \gets M[i][m] + M[n-i][m]$
            \EndIf
        \EndFor
        \For{$j \gets 1$ to $\lfloor m/2 \rfloor$} \Comment{Check all vertical cuts}
            \State \Call{COMPUTE-M}{$n, j, p, M$}
            \State \Call{COMPUTE-M}{$n, m-j, p, M$}
            \If{$M[n][j] + M[n][m-j] > \text{max}$}
                \State $max \gets M[n][j] + M[n][m-j]$
            \EndIf
        \EndFor
        \If{$p(n,m) > \text{max}$} \Comment{Sell the entire slab?}
            \State $max \gets P(n,m)$
        \EndIf
        \State $M[n][m] \gets \text{max}$
    \EndIf
    \State \Return $M[n][m]$
\EndFunction
\end{algorithmic}
\end{algorithm}

\section{Preprocessing}
The preprocessing step involves querying spot prices for all possible rectangle dimensions and storing the results in the array $p$. This step ensures constant-time access to spot prices during the algorithm's execution.

\section{Recurrence Relation}
The recurrence relation for the algorithm is given by:
\[
T\left(\frac{n}{2}, m\right) + T\left(n, \frac{m}{2}\right) + O\left(\frac{n}{2}\right) + O\left(\frac{m}{2}\right) + O(1)
\]
Where:
\begin{itemize}
    \item $T(n, m)$ is the time complexity of the algorithm for a problem of size $n \times m$.
    \item The recursive calls $T\left(\frac{n}{2}, m\right)$ and $T\left(n, \frac{m}{2}\right)$ represent the checks for horizontal and vertical cuts, respectively.
    \item $O\left(\frac{n}{2}\right)$ and $O\left(\frac{m}{2}\right)$ represent the work done in checking cuts.
    \item $O(1)$ represents the constant-time spot price query.
\end{itemize}

\section{Complexity Analysis}

\subsection{Time Complexity}
\begin{itemize}
    \item The recursive calls contribute $\log n + \log m$ levels, assuming each dimension is halved in each recursive call.
    \item The work done in checking cuts contributes $O(n) + O(m)$.
    \item The constant-time spot price query contributes $O(1)$.
\end{itemize}
\textbf{Overall Time Complexity:} $O((\log n + \log m) \cdot (n + m))$

\subsection{Space Complexity}
\begin{itemize}
    \item The space complexity is influenced by the recursive call stack, contributing $O(\log n + \log m)$.
    \item The memoization table $M$ contributes $O(n \cdot m)$.
\end{itemize}
\textbf{Overall Space Complexity:} $O(n \cdot m)$
\section{Assumptions}

In the context of the marble subdivision problem, the following assumptions are made:

\begin{enumerate}
    \item \textbf{Rectangular Slab Dimensions:} The algorithm assumes that the dimensions of the marble slab are positive integers, with the height denoted by $n$ centimeters and the width by $m$ centimeters. Negative or non-integer dimensions are not considered.

    \item \textbf{Spot Prices:} The algorithm assumes that spot prices for marble rectangles can be queried in $O(1)$ time for any given dimensions. This assumes efficient data structures or preprocessed information for spot prices.

    \item \textbf{Profit Maximization Objective:} The algorithm is designed with the objective of maximizing profit through the subdivision of the marble slab. Profit is defined based on the sum of spot prices for individual rectangular pieces.

    \item \textbf{Spot Price Dependence:} The algorithm does not make assumptions about the behavior of spot prices. It is agnostic to the specific characteristics of spot prices, including any dependencies on the dimensions of the rectangles.

    \item \textbf{Spot Price Precision:} The algorithm assumes that spot prices are provided with sufficient precision to differentiate between different configurations. This ensures accurate comparison and selection of profitable cuts.

    \item \textbf{Space and Time Complexity Model:} The complexity analysis assumes a standard computational model where basic arithmetic operations, array access, and spot price queries take constant time. Deviations from this model may impact the provided complexities.

    \item \textbf{Rectangular Cuts:} The algorithm assumes that cuts can only be made along the boundaries of the marble slab, resulting in rectangular pieces. Diagonal cuts or irregular shapes are not considered.

    \item \textbf{No Constraints on Cut Dimensions:} The algorithm does not impose constraints on the dimensions of the cut rectangles, except that they should be positive integers. It explores all possible combinations of cuts.

    \item \textbf{Initial Memoization Table Values:} The algorithm initializes a memoization table to store computed values. It assumes that the initial values in the table are set to a special marker (e.g., \texttt{NIL}) to identify uncomputed cases.

    \item \textbf{Threshold for Selling Entire Slab:} The algorithm considers selling the entire marble slab as a potential option. It determines whether selling the entire slab is more profitable than the subdivided pieces based on spot prices.

\end{enumerate}

These assumptions collectively provide a framework for the algorithm to address the marble subdivision problem while considering various dimensions and spot prices.
\section{C++ Code}
\begin{verbatim}
#include <iostream>
#include <vector>
#include <cmath>
#include <climits>
#include <chrono>
using namespace std;
using namespace chrono;


// Function to take input for spot prices
void inputSpotPrices(vector<vector<long long>>& prices, int n, int m) {
    cout << "Enter spot prices for the marble slab:" << endl;
    for (int i = 0; i < n; ++i) {
        cout << "Row " << i + 1 << ": ";
        for (int j = 0; j < m; ++j) {
            cin >> prices[i][j];
        }
    }
}

// Recursive function to compute maximum profit
long long computeM(int n, int m, vector<vector<long long>>& M, vector<vector<long long>>& prices) {
    if (M[n][m] != -1) {
        return M[n][m]; // Avoid recomputing
    } else if (n == 0 || m == 0) {
        M[n][m] = 0; // Base case
    } else {
        long long maxProfit = 0;

        // Check all horizontal cuts
        for (int i = 1; i <= floor(n / 2); ++i) {
            long long profit = computeM(i, m, M, prices) + computeM(n - i, m, M, prices);
            if (profit > maxProfit) {
                maxProfit = profit;
            }
        }

        // Check all vertical cuts
        for (int j = 1; j <= floor(m / 2); ++j) {
            long long profit = computeM(n, j, M, prices) + computeM(n, m - j, M, prices);
            if (profit > maxProfit) {
                maxProfit = profit;
            }
        }

        // Should we sell the entire slab?
        if (prices[n-1][m-1] > maxProfit) {
            maxProfit = prices[n-1][m-1];
        }

        M[n][m] = maxProfit;
    }

    return M[n][m];
}

int main() {
    int n, m; // Dimensions of the marble slab
    cout << "Enter the dimensions of the marble slab (n m): ";
    cin >> n >> m;

    // Initialize memoization table with NIL values
    vector<vector<long long>> memoTable(n + 1, vector<long long>(m + 1, -1));

    // Initialize spot prices vector
    vector<vector<long long>> spotPrices(n, vector<long long>(m, 0));

    // Call function to input spot prices
    inputSpotPrices(spotPrices, n, m);

    // Start measuring time
    auto start = high_resolution_clock::now();

    long long maxProfit = computeM(n, m, memoTable, spotPrices);

    // Stop measuring time
    auto stop = high_resolution_clock::now();
    auto duration = duration_cast<milliseconds>(stop - start);

    cout << "Maximum profit: " << maxProfit << endl;
    cout << "Time elapsed: " << duration.count() << " milliseconds" << endl;

    return 0;
}
\end{verbatim}
\section{Conclusion}
The algorithm efficiently subdivides the marble slab to maximize profit by exploring all possible cuts and considering spot prices. The time and space complexities provide insights into the algorithm's scalability and resource requirements.

\end{document}
